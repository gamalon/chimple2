\chapter{Introduction}

\begin{chapquote}{\scriptsize Montgomery Burns, \textit{The Simpsons} episode ``Last Exit to
    Springfield," to a room of 1000 monkeys}
```It was the best of times, it was the \emph{blurst} of times?' You stupid monkey!"
\end{chapquote}
%
\noindent
Chimple is a tool for performing inference on generative models.
The goal is to allow users to build complex models quickly: the user writes
simple code to specify a model, while Chimple performs automatic inference
under the hood.

Chimple is open source and can be accessed on Github
\begin{center}
\href{https://github.com/gamelanlabs/chimple2}{https://github.com/gamelanlabs/chimple2},
\end{center}
where any pull requests or bug reports should be sent.

\section{Installation}

\subsection{Requirements}
In order to use Chimple, you need:
\begin{itemize}
   \item Java 1.7+
   \item Matlab (if you want to use the Matlab wrapper).
\end{itemize}

Chimple has only been tested on Mac OS X and Windows operating systems
and for the interactive development environments (IDEs) Eclipse and IntelliJ.
We recommend using one of these environments after cloning the repository:
\begin{verbatim}
   git clone https://github.com/gamelanlabs/chimple2.git
\end{verbatim}

\subsection{Instructions for Eclipse}
For ease of use, we have included a Python script you can run to set your
classpath properly:

\subsection{Instructions for IntelliJ}

\section{Chimple Overview}

\subsection{Probabilistic Programs}
A probabilistic program is

\subsection{Chimple Programs}

In order to write a Chimple program, the user writes a program
corresponding to a generative model.
