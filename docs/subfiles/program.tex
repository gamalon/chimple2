\chapter{Chimple Program}

In this section, we focus on how to express a generative model as a Chimple
    program and detail the components involved.
Throughout this section, we reference terms from the Chimple language.
We summarize these terms in Table \ref{terms}.

We begin by presenting a simple example before delving into Chimple details.

\section{How to write a Chimple program}

\subsection{Random Coin Example}

Consider the following problem:
you have a biased coin, which you flip $N$ times and land as either heads or tails.
A generative model might be
\begin{enumerate}
   \item Draw a weight uniformly $\theta \sim \text{Beta}(1,1)$
   \item For $n \in \{1, \ldots, N\}$:
    \begin{enumerate}
        \item Flip the coin $X_n \sim \text{Bern}(\theta)$
    \end{enumerate}
\end{enumerate}
The generative model above can be expressed in Chimple as

\begin{center}
\begin{lstlisting}[style=customc]
    public Object run(Object... args)
    {
        // Generate a random weight for the coin
        double theta = chimpRand("weight");

        // Flip the coin N times
        for (int n = 0; n < N; n++)
            X[n] = chimpFlip(("X" + n, theta));

        return X;
    }
\end{lstlisting}
\end{center}

\label{priorsolver}
\subsection{Chimple program components}

In each Chimple program, there are a few main components:
\begin{enumerate}
    \item All programs must extend ChimpleProgram, an abstract class which
    holds many program-specific objects.
    \item All programs must implement the run function. This is where the generative model is
    implemented, as in the coin example.
    \item In the main function, you run the program with a solver.
        We will discuss this in more detail later.
        To run the program ``forward" and generate data, you can use the solver
        \texttt{PriorSolver}, as seen in the example below.
        This simply runs the code in the program's run function.
\end{enumerate}
%
\begin{center}
\begin{lstlisting}[style=customc]
    public class MyProgram extends ChimpleProgram
    {
        @Override
        public Object run(Object... args)
        {
            double[] data = new double[N];

            // Implement program here
            ...

            // Return the generated data
            return data;
        }

        public static void main(String[] args)
        {
            MyProgram program = new MyProgram();

            Solver s = PriorSolver();

            // TODO: fill in actual code
        }
    }
\end{lstlisting}
\end{center}
%
Note that the program above only specifies a generative model--no inference
    is performed with observed data, which we discuss in Chapter \ref{solvers}.

For a full demo, see \texttt{RandomCoin.java}. In the demos package of Chimple,
    we have included a suite of other demos.


\section{Chimple Monkeys}

In Chimple, we have included a library of \emph{monkeys}, or elemental random
primitives (ERPs), which are atomic random functions combined to create more complex functions.
In the example above, we use the ERP \texttt{chimpRand}, which generates a
value between 0 and 1 from Uniform distribution.

Most of the ERPs represent simple statistical distribution and their properties,
     also containing specific functions used for Chimple's inference engine.
     We will discuss ERPs more from the context of inference in Section \ref{solvers}.

We summarize the currently available ERPs in Table \ref{monkeys}.

\begin{figure}
   \centering
   \begin{tabular}{ccc}
    \toprule
       ERP Name & Arguments & Description \\
    \midrule
        chimpBeta & \\
        chimpDirichlet & \\
        chimpDiscrete & \\
        chimpFlip & \\
        chimpGamma & \\
        chimpNormal & \\
        chimpPermutation & \\
        chimpPoisson & \\
        chimpRand & \\
    \bottomrule
   \end{tabular}
   \caption{Current library of ERPs in Chimple.}
   \label{monkeys}
\end{figure}

\begin{figure}
\centering
\small
\begin{tabular}{ccc}
\toprule
   Chimple Name & Description & Associated Class(es) \\
\midrule
    Monkey & Abstract random variable, all monkeys extend this & Monkey.java \\
    Banana & Generic type for a Monkey  & Monkey.java \\
    Zookeeper & Bookkeeping for all Monkeys, used by the solver  & Zookeeper.java \\
    MonkeyCage & A program trace; stores Monkeys & MonkeyCage.java \\
    ChimpleProgram & Base program that all programs extend & ChimpleProgram.java \\
    Energy & Negative log likelihood & - \\
\bottomrule
\end{tabular}
\caption{Terms used in Chimple related to writing a probabilistic program.}
\label{terms}
\normalsize
\end{figure}
